%%%%%%%%%%%%%%%%%%%%%%%%%%%%%%%%%%%%%%%%%
% Stylish Curriculum Vitae
% LaTeX Template
% Version 1.0 (18/7/12)
%
% This template has been downloaded from:
% http://www.LaTeXTemplates.com
%
% Original author:
% Stefano (http://stefano.italians.nl/)
%
% IMPORTANT: THIS TEMPLATE NEEDS TO BE COMPILED WITH XeLaTeX
%
% License:
% CC BY-NC-SA 3.0 (http://creativecommons.org/licenses/by-nc-sa/3.0/)
%
%
%%%%%%%%%%%%%%%%%%%%%%%%%%%%%%%%%%%%%%%%%

\documentclass[a4paper, oneside, final]{scrartcl} % Paper options using the scrartcl class

\usepackage{scrpage2} % Provides headers and footers configuration
\usepackage{titlesec} % Allows creating custom \section's
\usepackage{marvosym} % Allows the use of symbols
\usepackage{tabularx,colortbl} % Advanced table configurations
\usepackage{fontspec} % Allows font customization
\usepackage{hyperref} % Allows hyperlinks in pdf documents -- Added by Jake
\usepackage{ifthen}

\newif\iftwopage
\ifthenelse{\equal{\detokenize{cv}}{\jobname}}{
  \twopagefalse
}{
  \twopagetrue
}

\definecolor{linkcolour}{rgb}{0.1,0.3,0.6}
\hypersetup
{
  colorlinks=true,
  linkcolor=linkcolour,
  urlcolor=linkcolour
}

\defaultfontfeatures{Mapping=tex-text}
\setmainfont{Buenard} % Main document font

\titleformat{\section}{\large\scshape\raggedright}{}{0em}{}[\titlerule] % Section formatting

\pagestyle{scrheadings} % Print the headers and footers on all pages

\addtolength{\voffset}{-0.5in} % Adjust the vertical offset - less whitespace at the top of the page
\addtolength{\textheight}{3cm} % Adjust the text height - less whitespace at the bottom of the page

\newcommand{\grey}{\rowcolor[gray]{.90}} % Custom highlighting for the work experience and education sections

%----------------------------------------------------------------------------------------
% �FOOTER SECTION
%----------------------------------------------------------------------------------------

\renewcommand{\headfont}{\normalfont\rmfamily\scshape} % Font settings for footer

\cofoot{
\addfontfeature{LetterSpace=0.0}\fontsize{12.5}{17}\selectfont % Letter spacing and font size

861 Beverley Road {\large\textperiodcentered} Hull {\large\textperiodcentered} United Kingdom HU6 9NH\\ % Your mailing address
{\Large\Letter} \href{mailto:jakebhumphrey@gmail.com}{jakebhumphrey@gmail.com} \ {\Large\Telefon} 07827 667139 % Your email address and phone number
}

%----------------------------------------------------------------------------------------

\begin{document}

\begin{center} % Center everything in the document

%----------------------------------------------------------------------------------------
% �HEADER SECTION
%----------------------------------------------------------------------------------------

{\addfontfeature{LetterSpace=0.0}\fontsize{36}{36}\selectfont\scshape Jake Humphrey} % Your name at the top

\vspace{1.0cm} % Extra whitespace after the large name at the top

%----------------------------------------------------------------------------------------
%	OBJECTIVE
%----------------------------------------------------------------------------------------
\iftwopage
\section{Objective}
\raggedright
I am currently looking for a position in the field of hardware design
using FPGAs, preferably in VHDL and/or with the Altera Quartus
toolchain. Opportunities for travel would be beneficial.

\fi
%----------------------------------------------------------------------------------------
%	WORK EXPERIENCE
%----------------------------------------------------------------------------------------

\section{Work Experience}

\begin{tabularx}{0.97\linewidth}{>{\raggedleft\scshape}p{2.5cm}X}
\grey Period & \textbf{September 2013 -- December 2013}\\
\grey Employer & \textbf{EEE Department} \hfill Imperial College London\\
\grey Job Title & \textbf{Teaching Assistant}\\

& Worked under the first-year Software Engineering lecturer to assist
students in lab sessions. Responded to student queries and gave them
advice and guidance.
\end{tabularx}

\begin{tabularx}{0.97\linewidth}{>{\raggedleft\scshape}p{2.5cm}X}
\grey Period & \textbf{July 2013 -- September 2013}\\
\grey Employer & \textbf{Altera} \hfill Imperial College London\\
\grey Job Title & \textbf{Intern}\\

& Worked in the High-Performance Embedded and Distributed Systems
laboratory at Imperial College. Ported a C program to a Java-like
hardware description language to run on an Altera FPGA.
\end{tabularx}

%% \vspace{12pt}

%% \begin{tabularx}{0.97\linewidth}{>{\raggedleft\scshape}p{2.5cm}X}
%% \grey Period & \textbf{June 2009}\\
%% \grey Employer & \textbf{BP Saltend} \hfill Kingston upon Hull, East Yorkshire\\
%% \grey Job Title & \textbf{Unpaid Intern}\\
%% %\grey Languages & \textbf{J2EE}\\

%% & Worked with the order-to-sales team in order to gain experience
%% working as part of a team in a corporate environment. Communicated
%% with customers and completed order forms.
%% \end{tabularx}

%\vspace{12pt}
%
%\begin{tabularx}{0.97\linewidth}{>{\raggedleft\scshape}p{2.5cm}X}
%\grey Period & \textbf{March 2009 --- August 2010 (Part Time)}\\
%\grey Employer & \textbf{Buy More} \hfill New York, USA\\
%\grey Job Title & \textbf{Supermarket Clerk}\\
%� � � �& Lorem ipsum dolor sit amet, consectetur adipiscing elit. Donec et auctor neque. Nullam ultricies sem sit amet magna tristique imperdiet.
%\end{tabularx}

%----------------------------------------------------------------------------------------
%	EDUCATION
%----------------------------------------------------------------------------------------

\section{Education}

\begin{tabularx}{0.97\linewidth}{>{\raggedleft\scshape}p{2.5cm}X}
\grey Period & \textbf{October 2011 -- July 2015}\\
\grey Qualification & \textbf{MEng Electronic and Information Engineering}\\
\grey Rank & \textbf{First Class Honours}\\
\grey Institute & \textbf{Imperial College London}  (Years~1--3) \hfill London, UK\\
\grey & \textbf{Telecom ParisTech} (Year~4) \hfill Paris, France\\
%&\hyperref[tab:degree_details]{Details}
\iftwopage
& Final year abroad as part of the Erasmus scheme.
\fi
\end{tabularx}

\iftwopage

\begin{tabularx}{0.97\linewidth}{>{\raggedleft\scshape}p{2.5cm}X}
\grey Period & \textbf{September 2009 -- July 2011}\\
\grey Qualification & \textbf{A-Levels}\\
%\grey Rank & \textbf{First Class Honours}\\
\grey Institute & \textbf{Hymers College} \hfill Kingston-upon-Hull, UK\\
%&\hyperref[tab:degree_details]{Details}
& Mathematics A*, Physics A*, Further Maths A, Electronics A
\end{tabularx}

\fi
%----------------------------------------------------------------------------------------
%       PROJECTS
%----------------------------------------------------------------------------------------

\section{Projects}

\iftwopage

\setlength{\parskip}{0.5em}

\begin{tabularx}{0.97\linewidth}{>{\raggedleft\scshape}p{2.5cm}X}
\grey Period & \textbf{May 2012}\\
\grey Setting & \textbf{First-Year Project} \hfill Imperial College London\\
\grey Project Title & \textbf{Spacewar!}\\
\end{tabularx}

Task: With a partner, create a program in Handel-C using the DK Design
Suite to run on an Altera FPGA which demonstrates Image Processing
capabilities. Other details are left to the students.

We decided to remake "Spacewar!"​, a very early video game developed
for the PDP-1 in 1962. The image processing came from the fact that we
stored certain game images as sprites, and the sprites representing
the players' ships were rotated on-the-fly before being drawn to the
screen.

We ran into some problems, such as not being able to reach the fixed
framerate target, which we solved using our Computer Architecture
knowledge to create a very rudimentary graphics pipeline.

For this project we received a departmental award for best in the
year.

\begin{tabularx}{0.97\linewidth}{>{\raggedleft\scshape}p{2.5cm}X}
\grey Period & \textbf{July 2013 -- September 2013}\\
\grey Setting & \textbf{Altera Internship} \hfill Imperial College London\\
\grey Project Title & \textbf{Air Traffic Management Optimisation}\\
\end{tabularx}

For this project we were given a C program which used Sequential Monte
Carlo simulations to provide travel vectors for aircraft. Our task was
to analyse the code and see how it could be optimised by running on an
FPGA. The FPGA was provided by Altera, and we were to use Maxeler's
Maxcompiler Suite for hardware design.

The MaxCompiler suite facilitates writing programs for CPUs with FPGA
coprocessors, and incorporates a C compiler along with a Java-like
language for hardware (kernel) design.

Although documentation and support for MaxCompiler were sparse, we
were able to succeed in designing a kernel to accelerate the most
CPU-intensive parts of the code.

\begin{tabularx}{0.97\linewidth}{>{\raggedleft\scshape}p{2.5cm}X}
\grey Period & \textbf{March 2014}\\
\grey Setting & \textbf{VHDL and Logic Synthesis course} \hfill Imperial College London\\
\grey Project Title & \textbf{VHDL Graphics Driver}\\
\end{tabularx}

Task: To design, implement, and test a logic circuit capable of
interpreting a predefined list of commands and colouring appropriate
locations in a pixel buffer. The commands allowed the user to draw
lines and boxes with a black, white, or inverting brush. The pixel
buffer was simply a contiguous area of RAM with a 0 representing a
white pixel and 1 representing black.

The design was split into two parts: the Draw Block which interpreted
commands and simplified them into a stream of "x, y, colour" commands
to the RAM Control Block, which took these commands and made the
appropriate changes in memory.

We also designed a suite of tests to validate the design, exploring
exhaustive, random, targeted, and constrained random testing
methodologies.

\begin{tabularx}{0.97\linewidth}{>{\raggedleft\scshape}p{2.5cm}X}
\grey Period & \textbf{May 2014 -- July 2014}\\
\grey Setting & \textbf{3rd Year Industry Group Project} \hfill Imperial College London\\
\grey Project Title & \textbf{VHDL Floating-Point Unit}\\
\end{tabularx}

As a group of six, design, implement, and test a VHDL Floating-Point
Unit (FPU). This project was proposed by a team of engineers from the
PowerVR Group at Imagination Technologies, and was carried out under
the scope of the 3rd Year Project in the Electrical and Electronic
Engineering department at Imperial College.

The six members split into four groups to handle four different
tasks. While the Design Engineers wrote the VHDL code, the Architect
wrote bit-equivalent C++ code. This was intended to mimic
Imagination's own workflow, as they often ship C++ code for clients to
test.

In addition, the Verification Engineers created testbenches to
robustly identify flaws in the design, while the Application Engineer
ensured that the design's projected area, frequency, and energy usages
fell within expected parameters.

\begin{tabularx}{0.97\linewidth}{>{\raggedleft\scshape}p{2.5cm}X}
\grey Period & \textbf{December 2014 -- June 2015}\\
\grey Setting & \textbf{Master's Thesis} \hfill Telecom Paristech\\
\grey Project Title & \textbf{Hardware-based Buffer Overflow Protection}\\
\end{tabularx}

My Master's Thesis explored a hardware-based approach to preventing
buffer overflow attacks. Using an OpenRISC processor (written in
Verilog HDL) as a base, I added several hardware registers to the
Memory Management Unit (MMU) to enable invalidation of sections of
memory. Inspired by canary values, the CPU was wired to raise an
exception when invalid memory was accessed, allowing OS intervention.

\else
\begin{itemize}
  \setlength{\itemsep}{0.3em}%
  \setlength{\parskip}{0em}
\item
First-Year Project. Worked with a partner to design a game in Handel-C
running on an FPGA which was able to rotate sprite images and draw
them to a VGA display.

\item
Second-year Language Processors coursework. Designed and implemented a
C-to-ARM assembler using Flex and Bison.

\item
Third year group project sponsored by Imagination Technologies. Worked
as part of a 6-person team to design, implement, and test a
Floating-Point Unit in VHDL

\item
Master's Thesis. Modified the Data MMU of the OpenRISC processor to
implement a new system providing protection against buffer overflow
attacks.
\end{itemize}

\fi
%----------------------------------------------------------------------------------------
%	SKILLS
%----------------------------------------------------------------------------------------

\section{Proficiencies}

%% \begin{tabular}{ @{} >{\bfseries}r @{\hspace{6ex}} l }
%% Encountered & Handel-C, Bash, Zsh, Verilog HDL, Git, Mercurial, \\
%% & Javascript, Java, Blender, PHP \\
%% Use often & Microsoft Windows, Microsoft Office, Matlab, \LaTeX, \\
%% & Lua, Quartus, HTML, CSS \\
%% Preferred & GNU/Linux, Arch Linux, C, C++, Emacs, Python
%% \end{tabular}

\begin{flushleft}
Windows and Unix-based systems, Office software, Shell scripting, C,
C++, Python, Lua, Java, Verilog, VHDL, Quartus, ModelSim, Synplify,
Matlab, \LaTeX, Git.
\end{flushleft}

\iftwopage

\section{Interests}

\begin{flushleft}
Graphics, Artificial Intelligence, FPGA Design, Network Security,
Kendo, Japanese Culture, Video Games
\end{flushleft}

\fi

%----------------------------------------------------------------------------------------

\end{center}

\section{References}
\centering
\begin{tabular*}{0.9\textwidth}{@{\extracolsep{\fill} }l l}
Dr. Moez Draeif & Guillaume Duc\\
Personal Tutor at Imperial College & Researcher at Télécom ParisTech\\
\href{mailto:m.draief@imperial.ac.uk}{m.draief@imperial.ac.uk}
& \href{mailto:guillaume.duc@telecom-paristech.fr}{guillaume.duc@telecom-paristech.fr}\\
\end{tabular*}

\end{document}
